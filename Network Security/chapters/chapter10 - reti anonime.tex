\setchapterpreamble[u]{\margintoc}
\chapter{Reti anonime}
\labch{chapter10}

Privacy sulle reti pubbliche:
\begin{itemize}
    \item Internet è disegnato come una rete pubblica: le macchine potrebbero vedere il traffico delle altre e i router vedono tutto il traffico che passa loro attraverso;
	\item Le info dul routing sono pubbliche: è facile capire chi sta comunicando con chi osservando l'header IP;
	\item La crittografia non nasconde le identità: la crittografia nasconde il payload, ma non il routing. Anche IPsec (tunnel mode/ESP) rivela gli indirizzi IP al gateway IPsec.
\end{itemize}

Applicazioni dell'anonimità:
\begin{itemize}
    \item Privacy;
	\item Email irrintracciabili;
	\item  Comunicazioni segrete su reti pubbliche;
	\item Denaro digitale (acquisti online non linkabili all'identità del compratore);
	\item Voto elettronico anonimo;
	\item Pubblicazioni contro la censura.
\end{itemize}
	
Cos'è l'anonimità:
\begin{itemize}
    \item Stato in cui non si è identificabile all'interno di un insieme di soggetti (nascono le mie attività tra altri simile; non posso essere anonimo da solo);
	\item Incollegabilità tra azione e identità;
	\item Inosservabilità (l'osservatore non riesce a dire se l'azione è stata fatta o no, difficile da ottenere).
\end{itemize}

Attacchi all'anonimità:
\begin{itemize}
    \item Analisi passiva del traffico: cerco di capire dal traffico della rete chi parla con chi;
	\item Analisi del traffico attiva: inietto pacchetti;
	\item Compromissione dei nodi della rete (router): non è ovvio capire quale nodo è stata compromesso.
\end{itemize}

L'anonimato è il migliore quando il servizio di anonimizzazione attira molti utenti (TOR).
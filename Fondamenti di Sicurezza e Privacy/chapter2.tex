\chapter{Cyber Kill Chain}
\label{chapter2}

\paragraph{Threat intelligence} Analisi dei possibili attacchi informatici che possono colpire l'organizzazione. Determinare le fasi di un attacco e le sue tecniche permette di fare una scelta mirata sulle misure di protezione da adottare per prevenirlo o ridurne l'impatto. \\

\section{Fasi della cyber kill chain}
La cyber kill chain, in generale, descrive le fasi in cui può essere suddiviso un attacco:
\begin{enumerate}
    \item Reconnaissance: decidere e ottenere info sul target. La reconnaissance può essere:
\begin{itemize}
    \item Attiva: raccolgo info interagendo col target.
        \begin{itemize}
            \item nmap per creare una mappa della rete dell'obiettivo (porte disponibili, porte aperte, servizi associati); 
	        \item port scanning;
	          \item vulnerability scanner (es: nessus);
	        \item Linkedin per cercare ruoli specifici nell'organizzazione.
        \end{itemize}
    \item Passiva: raccolgo info senza interagire col soggetto.
	      \begin{itemize}
	          \item whois permette di raccogliere info su un dominio o specifico IP: quando è stato creato; data rilasco e scadenza certificato; chi è l'amministratore di sistema; contatto tecnico per rinnovo dominio;
	          \item Shodan: porte aperte; servizi accessibili; lista vulnerabilità;
	        \item Social media;
	        \item mantego.
	      \end{itemize}
\end{itemize}

	
\item Weaponization: trovare o creare l'attacco che exploita una vulnerabilità. Posso usare:
\begin{itemize}
    \item Metasploit per creare un attacco che sfrutta vulnerabilità. Fornisce codice già pronto per sfrittare vulnerabilità;
    \item Exploit DB;
    \item Vell Framework;
    \item Aircrack;
    \item SQL map;
    \item Cobal strike, tool legittimo a pagamento usato per penetration testing. Ha struttura modulare.
\end{itemize} 

\item Deliver: decido in che modo consegnare l'exploit. Posso sfruttare: siti web, social media, email, USB, ...

\item Exploitation: exploito una vulnerabilità. Posso usare:
\begin{itemize}
    \item  SQL injection;
    \item Buffer overflow;
    \item Malware;
    \item Javascript hijacking;
    \item User exploitation;
    \item Compromissione sito legittimo (inserimento script o sfruttamento vulnerabilità framewok usati nel sito).
\end{itemize}

\item Installation: mantengo persistente l'attacco all'interno dell'ambiente. Posso usare:
\begin{itemize}
    \item DDL hijacking;
\item Meterpreter;
\item Remote Access Trojan;
\item Registry chenges;
\item PowerShell commands.
\end{itemize}

\item Command and Control (C2): stabilisco un canale di C2 conl'attaccante per manipolare da remoto la vittima.

\item Actions and Objectives: eseguo azioni per reggiungere il goal originario. Posso:
\begin{itemize}
    \item Raccogliere credenziali utente;
\item Scalare privilegi;
\item Reconnaissance interna;
\item Movimento nell'ambiente;
\item Recccogliere dati;
\item Distruggere sistemi;
\item Riscrivere o corrompere dati;
\end{itemize}
\end{enumerate}

\section{Trickbot}
Torjan avanzato che si diffonde principalmente tramite campagne di spearphishing usando mail che contengono allegati malevoli o link che eseguono sw malevoli. Gli attacchi recenti usano mail di phishing.
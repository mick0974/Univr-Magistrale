\chapter{Protocollo OAuth}
Vedremo ora degli scenari in cui è l'utente che decide se garantire o meno l'accesso alle sue risorse (servizi cloud come Drive, dispositivi IoT, ...).

Per gestire queste situazioni è necessario un protocollo che garantisca che solo le applicazioni decise dall'utente possano accedere a queste risorse (risorse ospitate sul web, dispositivi smart da controllare, ...).

Il protocollo usato è OAuth (Open Autorization), che è un protocollo standard che consente ad applicazioni di terze parti di accedere a risorse protette hostate su un server HTTP (es: schermata di richiesta di accesso ad info dell'account Google nei giochi per cellulare). L'accesso viene garantisco se approvato dall'utente.

Il protocollo prevede la presenza di un Authorization Server che rilascia all'applicazione un access token quando l'utente garantisce l'accesso alle proprie risorse usando quell'app.
\\

\noindent Gli attori principali sono quanto:
\begin{itemize}
    \item Owner delle risorse, che è l'entità che può garantire l'accesso alla risorsa;
    \item Resource Server, che è il server che ospita le risorse dell'utente;
    \item Authorization Server, che è il server che rilascia l'access token al client dopo aver autenticato il proprietario della risorsa e ottenuto l'autorizzazione;
    \item Client, che è l'app di terze parti che l'utente usa per accedere alle risorse.
\end{itemize}

\noindent Scenario tipico: voglio accedere a Spotify tramite le credenziali Facebook. Le risorse protette sono le mie credenziali Facebook.
\\

\noindent Il protocollo supporta 5 casi differenti:
\begin{enumerate}
    \item \textbf{Authorization Code Grant Flow};
    \item \textbf{Authorization Code Grant Flow with PCKE};
    \item \textbf{Resource Owner Password};
    \item \textbf{Client Credential};
    \item \textbf{Device Flow}.
\end{enumerate}

\section{Authorization Code Grant Flow}
Usata quando si accede a un'applicazione usando un account Google o Facebook.

\subsection{Flusso}
Il client reindirizza l'utente al server di autorizzazione con i seguenti parametri nella stringa di query:
\begin{itemize}
    \item \textbf{response\_type} con il valore \textbf{code};
    \item \textbf{client\_id} con l'identificatore del client;
    \item \textbf{redirect\_uri} con l'URI di reindirizzamento del client. Questo parametro è facoltativo, ma se non viene inviato l'utente verrà reindirizzato a un URI di reindirizzamento preregistrato.
    \item \textbf{scope} con un elenco di scope;
    \item \textbf{state} con una stringa casuale usata per prevenire attacchi (token CSRF). 
\end{itemize}

\noindent Tutti questi parametri verranno convalidati dal server di autorizzazione. All'utente verrà quindi chiesto di accedere al server di autorizzazione e approvare il client.

Se l'utente approva il client, verrà reindirizzato dal server di autorizzazione al client (in particolare all'URI di reindirizzamento) con i seguenti parametri nella stringa di query:
\begin{itemize}
    \item \textbf{code} con il codice di autorizzazione;
    \item \textbf{state} con il parametro \textbf{state} inviato nella richiesta originale. È consigliabile confrontare questo valore con il valore archiviato nella sessione dell'utente per assicurarsi che il codice di autorizzazione ottenuto risponda alle richieste effettuate da questo client anziché da un'altra applicazione client.
\end{itemize}

\subsection{Flusso (pt. 2)}
Il client invierà ora una richiesta POST al server di autorizzazione con i seguenti parametri:
\begin{itemize}
    \item \textbf{grant\_type} con il valore di \textbf{authorization\_code};
    \item \textbf{client\_id} con l'identificatore del client;
    \item \textbf{client\_secret} con il segreto client;
    \item \textbf{redirect\_uri} con lo stesso URI di reindirizzamento a cui l'utente è stato reindirizzato;
    \item \textbf{code} con il codice di autorizzazione dalla stringa di query.
\end{itemize}

\noindent Il server di autorizzazione risponderà con un oggetto JSON contenente le seguenti proprietà:
\begin{itemize}
    \item \textbf{token\_type} questa sarà di solito la parola "Bearer" (per indicare un token al portatore);
    \item \textbf{expires\_in} con un numero intero che rappresenta il TTL del token di accesso (ovvero quando il token scadrà);
    \item \textbf{access\_token} con token di accesso stesso
    \item \textbf{refresh\_token} con un refresh token che può essere utilizzato per acquisire un nuovo token di accesso alla scadenza dell'originale.
\end{itemize}

\noindent A questo punto il client presenta l'access token all'Authorization Server, che fornisce la risorsa all'app.

Questo processo è applicato a tutte le app native (es: su cellulare) o web app in cui il fornitore del client è diverso dal fornitore della risorsa.

\section{Authorization Code Flow with PCKE}
Simile all'Authorization Code Flow, ma presenta due differenze fondamentali:
\begin{itemize}
    \item È usato in client basati su user-agent (ad esempio app Web a pagina singola) che non possono mantenere segreto un client perché tutto il codice dell'applicazione e l'archiviazione sono facilmente accessibili;
    \item L'authorization server ritorna direttamente l'access token piuttosto che in codice di autorizzazione da scambiare per l'access token.
\end{itemize}

\subsection{Flusso}
Il client reindirizza l'utente al server di autorizzazione con i seguenti parametri nella stringa di query:
\begin{itemize}
    \item \textbf{response\_type} con il valore "token";
    \item \textbf{client\_id} con l'identificatore del client;
    \item \textbf{redirect\_uri} con l'URI di reindirizzamento del client. Questo parametro è facoltativo, ma se non viene inviato l'utente verrà reindirizzato a un URI di reindirizzamento preregistrato;
    \item \textbf{scope} con un elenco di scope delimitato da spazi;
    \item \textbf{state} con un token CSRF.
\end{itemize}

\noindent Tutti questi parametri verranno convalidati dal server di autorizzazione. All'utente verrà quindi chiesto di accedere al server di autorizzazione e approvare il client.

Se l'utente approva il client, verrà reindirizzato al server di autorizzazione con i seguenti parametri nella stringa di query:
\begin{itemize}
    \item \textbf{token\_type} con il valore "Bearer";
    \item \textbf{expires\_in} con un numero intero che rappresenta il TTL del token di accesso;
    \item \textbf{access\_token} con l'access token stesso;
    \item \textbf{state} con il parametro state inviato nella richiesta originale.
\end{itemize}

\noindent Questa soluzione non restituisce un refresh token perché il browser non ha mezzi per mantenerlo privato.

\section{Resource Owner Password}
Questa soluzione è usata client first party affidabili sia sul Web che nelle applicazioni per dispositivi nativi.

\subsection{Flusso}
Il client chiede all'utente le credenziali di autorizzazione (di solito un nome utente e una password). Il client invia quindi una richiesta POST con i seguenti parametri del corpo al server di autorizzazione:
\begin{itemize}
    \item \textbf{grant\_type} con il valore "password";
    \item \textbf{client\_id} con l'ID del client;
    \item \textbf{client\_secret} con il segreto del client;
    \item \textbf{scope} con un elenco delimitato delle autorizzazioni richieste;
    \item \textbf{username} con il nome utente dell'utente;
    \item \textbf{password} con la password dell'utente.
\end{itemize}

\noindent Il server di autorizzazione risponderà con un oggetto JSON contenente le seguenti proprietà:
\begin{itemize}
    \item \textbf{token\_type} con il valore "Bearer";
    \item \textbf{expires\_in} con un numero intero che rappresenta il TTL del token di accesso;
    \item \textbf{access\_token} con il token di accesso stesso;
    \item \textbf{refresh\_token} con un refresh token che può essere utilizzato per acquisire un nuovo token di accesso alla scadenza dell'originale.
\end{itemize}

\section{Client Credential}
La più semplice di tutte le concessioni OAuth 2.0, questa soluzione è adatta per l'autenticazione da computer a computer in cui non è richiesta l'autorizzazione di un utente specifico per accedere ai dati.

\subsection{Flusso}
Il client invia una richiesta POST con i seguenti parametri al server di autorizzazione:
\begin{itemize}
    \item \textbf{grant\_type} con il valore "client\_credentials";
    \item \textbf{client\_id} con l'ID del cliente;
    \item \textbf{client\_secret} con il segreto del cliente;
    \item \textbf{scope}.
\end{itemize}

\noindent Il server di autorizzazione risponderà con un oggetto JSON contenente le seguenti proprietà:
\begin{itemize}
    \item \textbf{token\_type} con il valore "Bearer";
    \item \textbf{expires\_in} con un numero intero che rappresenta il TTL del token di accesso;
    \item \textbf{access\_token} con il token di accesso stesso.
\end{itemize}

\section{Device flow}
Vedere PDF.



































 
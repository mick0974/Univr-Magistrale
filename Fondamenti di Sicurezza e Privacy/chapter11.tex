\chapter{Risk Management}

Il risk managment è il processo che consiste nell'identificare, analizzare e valutare il rischio. Si tratta dell'unico modo per assicurare che i controlli di cyber security scelti siano appropriati ai rischi che l'organizzazione affronta. Senza rick assessment ad informare le scelte di cyber security, c'è il rischio di sprecare tempo, sforzo e risorse. Ha poco senso infatti implementare misure di sicurezza contro minacce improbabili che accadano o che abbiano impatto materiale limitato sull'organizzazione, senza contare che vi è il rischio di sottostimare o ignorare rischi che potrebbero effettivamente colpire l'organizzazione. 

Le organizzazioni devono decidere quanto tempo e denaro spendere proteggere la loro tecnologia e i loro servizi. Uno degli obiettivi principali della gestione del rischio è quello di informare e migliorare queste decisioni. In base al settore in cui l'azienda opera, il risk management può essere un obbligo. Ad esempio, un'organizzazione che vuole la certificazione ISO 27001, avere una strategia di risk management è uno dei requisiti chiave (è richiesto anche da GDPR).
\\

\noindent Un processo di risk management si compone di più fasi:
\begin{itemize}
    \item Nella fase iniziale, l'azienda che lo mette in atto deve definire la strategia per il risk management, dove vengono definiti la metodologia usata dall'azienda, le componenti essenziali dell'organizzazione che ne devono essere soggette (es: informazione, processi che elaborano i servizi), chi sono gli stakeholders del sistema;
    \item Nella fase cardine, quella di risk assessment, si identificano, analizzano e valutano i rischi;
    \item Nella fase di Risk Treatment l'organizzazione, per ciascun dei rischi individuati, vengono identificate le misure di protezione. In questa fase non bisogna tenere presente solo il rischio però, ma anche le risorse disponibili (economiche, personale, ...). Alla fine del processo esisterà sempre un rischio minimo che non è stato mitigato. L'azienda dovrà quindi decidere se accettare il rischio, per continuare a fornire i suoi servizi, monitorare il rischio in attesa di cambiamenti oppure trasferire il rischio a terzi (assicurazione);
    \item I rischi identificati vanno infine comunicati agli stakeholder del sistema;
    \item Nella fase di monitor and review le misure di protezione selezionate e implementatevengono monitorate per garantire che forniscano sempre un livello di protezione adeguato rispetto agli attacchi identificati nella fase di risk assessment. Questo può essere fatto con penetration testing o valutando le performance delle misure adottate intervistando gli stakeholders. In base al risultato di questa fase, potrebbe essere necessario ripartire da capo. L'attività di revisione del rischio può anche essere fatta qualora l'azienda decide di introdurre nuove funzionalità/sistemi o raccogliere nuove informazioni. Ad esempio, se l'azienda decide di spostare i dati dei clienti dai server locale a un servizio cloud, è necessario rifare il processo da capo
\end{itemize}

\noindent Il processo di risk management è un processo ciclico

\section{}

Il NISP fornisce uno standard per le fase di risk management viste. Il vantaggio di usare gli standard del NISP è che sono open-surce.

\noindent Tra gli standard che fornisce troviamo:
\begin{itemize}
    \item Lo standard 800-39 che ci dà un'introduzione generale ai concetti alla base del processo di information management;
    \item Lo standard 800-37 che ci descrive come implementare un programma di risk management;
    \item Lo standard 800-30 che ci descrive il processo di risk assessment;
    \item Lo standard 800-53 che riguarda la fase di risk treatment;
    \item Lo standard 800-53a che descrive come valutare l'efficacia dei security controls identificati nella fase di risk treatment;
    \item Lo standard 800-39 che descrive come classificare le informazioni per il risk assessment, se il focus dell'azienda consiste in quelle;
\end{itemize}

\noindent Oltre gli standard del NIST ne esistono altri:
\begin{itemize}
    \item ISO/IEC IS 27005 che fornisce un processo generale per il risk management, non legato specificatamente alla sicurezza;
    \item ISO 31000, un altro standard generale;
    \item Altre metodologie nazionali.
\end{itemize}

\noindent Esistono anche metodologie legate solo al risk assessment: OCTAVE, CORAS (basata solo su grafici),  EBIOS, CRAMM, ...


\noindent Quando parliamo di rischio dobbiamo definire i fattori che andiamo ad usare per quantificare il rischio. Quasi tutte le metodologie parlano di rischio rispetto all'asset. Gli asset contengono vulnerabilità. I threat actor iniziano uno o più threat event che vanno a creare un threat scenario, un attacco che consiste di più fasi. 

Il rischio è dato da due componenti: la probabilità che avvenga e l'impatto che questo potrebbe avere su uno o più asset dell'organizzazione. 

Vediamo uno scenario generico: il rischio si materializza quando abbiamo un threat agent che inizia un attacco sfruttando delle vulnerabilità che risulta in un impatto negativo per l'organizzazione. Il rischio viene dato dall'impatto negativo e dalla likelyhood. Questa è composta dalla probabilità che l'attacco avvenga e dalla probabilità che questo ha di impattare. 





Identificare i rischi

Supponiamo di avere Poste Italiane che permette di eseguire operazione di online bancking, sia tramite app online che app da cellulare. Le informazione vengono memorizzate da un online bancking service, a cui le due app accedono. Per accedere ai servizi, i clienti devono utilizzare un username e una password.

Per identificare i rischi di questo scenario usiamo il risk assessment definito nello standarn NIST 800-30. Lo standard si compone di 4 fase:
\begin{enumerate}
    \item Raccolta di tutti documenti necessari per fare l'analisi del rischio;
    \item Identificazione dei rischi e valutazione della loro severità in base alla likelyhood;
    \item Comunicazione dei risultati del risk assessment al CEO, CESO e manager, ceh decidono quelai rischi mitigare;
    \item Revisione periodica del risk assessment, per adeguarlo ai cambiamoenti.
\end{enumerate}

\noindent Nella fase 1bisogna determinare perchè si sta facendo l'assessment. Esistono due possibilità:
\begin{itemize}
    \item Si tratta del primo assessment;
    \item 
\end{itemize}

23:00

































